\documentclass[c_worksheet.tex]{subfiles}

\begin{document}
  
\chapter{Eigene Datentypen / Structs}

\section{Zusammengesetze Datentypen erstellen}

Die Standard Datentypen wie \textbf{int, float, double, char}... kennen wir schon. Wenn man sich neue Variablen von diesen Typen erstellen möchte schreibt man:
\begin{lstlisting}
int i;   //erstellt eine Variable vom Typ int
char c;  //        .    .    .           char
float f; //        .    .    .           float
\end{lstlisting}

Wenn man nun Dinge aus der realen Welt in Programmcode repräsentieren möchte, überlegt man sich aus welchen Variablen diese Dinge aufgebaut sind.\\
z.B.: \textit{Ein Datum besteht aus drei Zahlen. Jahr, Monat und Tag. Also sollte ich mir drei Variablen erstellen.} \\

Der Code dazu könnte so aussehen:
\begin{lstlisting}
int main(void) {
    //3 Variablen fuer ein Datum erstellen
    int zeitpunkt1_jahr;
    int zeitpunkt1_monat;
    int zeitpunkt1_tag;

    //3 Variablen fuer ein anderes Datum erstellen
    int zeitpuntk2_jahr;
    int zeitpuntk2_monat;
    int zeitpunkt2_tag;

    //alle Variablen mit Werten fuellen
    zeitpunkt1_jahr = 2015;
    zeitpunkt1_monat = 9;
    zeitpunkt1_tag = 15;
    zeitpunkt2_jahr = 2016;
    zeitpunkt2_monat = 3;
    zeitpunkt2_tag = 26;

    return 0;
}
\end{lstlisting}

Weil das sehr schnell sehr unübersichtlich wird (am Ende hat man tausende Variablen) darf man sich eigene Datentypen erstellen, die aus Komponenten aufgebaut sind.
Diese selbst erstellten Datentypen tragen den Namen \textbf{struct} (weil der Code durch sie strukturierter wird).

\begin{lstlisting}
//hier wird kein Befehl ausgefuehrt
//hier wird ein neuer Datentyp mit dem Name 'struct Datum' erstellt
struct Datum {   //der neue Datentyp hat 3 Komponenten
    int jahr;
    int monat;
    int tag;
};

int main(void) {
    //2 Variablen vom Typ 'struct Datum' erstellen
    struct Datum zeitpunkt1;
    struct Datum zeitpunkt2;

    //und mit Werten fuellen
    zeitpunkt1.jahr = 2015;
    zeitpunkt1.monat = 9;
    zeitpunkt1.tag = 15;

    zeitpunkt2.jahr = 2016;
    zeitpunkt2.monat = 3;
    zeitpunkt2.tag = 26

    return 0;
}
\end{lstlisting}

In den Zeilen 11 und 12 werden Variablen mit den Namen \textit{zeitpunkt1} und \textit{zeitpunkt2} erstellt. Mit dem \textbf{Punkt-Operator} kann man dann auf die Komponenten dieser Variablen zugreifen. Diese kann man genauso benutzen wie einzelne Variablen. Man kann Werte in sie hineinschreiben und Werte aus ihnen herauslesen (siehe 2.3)

\begin{lstlisting}
zeitpunkt1.jahr = 2015;
zeitpunkt2.jahr = zeitpunkt1.jahr;
\end{lstlisting}

Es gibt noch eine kürzere Schreibweise um das struct bei der Initialisierung direkt mit Werte zu füllen. Dabei ist es wichtig die Reihenfolge, in der die Variablen des structs definiert worden sind, zu beachten.

\lstinputlisting{CodeSnippets/EigeneDatentypen/direkte_zuweisung.c}

Das ist vor allem hilfreich wenn man \emph{strings} in der Form von \emph{character arrays} in seinem struct hat. Eine direkte Zuweisung von Text funktioniert nämlich nur direkt bei der Initialisierung. Hierbei empfiehlt es sich \emph{char pointer} zu verwenden. Folgendes Beispiel veranschaulicht dies.

\lstinputlisting{CodeSnippets/EigeneDatentypen/char_arrays.c} 


Will man einen neuen Datentyp erstellen schreibt man allgemein:
\begin{lstlisting}
struct <name> {
    <Typ von Komponente 1> <Name von Komponente 1>;
    <Typ von Komponente 2> <Name von Komponente 2>;
      .
      .
      .
};
\end{lstlisting}

\section{Datentypen ineinander schachteln}

Variablen eines selbst erstellten Datentyps können auch wieder Komponenten eines anderen Datentyps sein.

\begin{lstlisting}
#include <stdio.h>

//erstellt Datentypen mit Namen 'struct Datum'
struct Datum { // 'struct Datum' besteht aus 3 Komponenten
    int jahr;
    int monat;
    int tag;
};

/*erstellt Datentypen mit Namen 'struct Name'
  beide Komponenten sind von Typ 'Pointer auf char'  */
struct Name { 
    char *vorname;   //Speicheradresse an der der Vorname liegt
    char *nachname;  //Speicheradresse an der der Nachname liegt
};

/*ein Student hat 3 Komponenten
  Komponente 'name' ist vom Typ 'struct Name'
  Komponente 'geburtsdatum' ist vom Typ 'struct Datum'
  Komponente 'matrikelNummer' ist vom Typ 'int' */
struct Student {
    struct Name name;
    struct Datum geburtsdatum; 
    int matrikelNummer; 
};

int main(void) {
    struct Student Max;
    Max.matrikelNummer = 335689;
    Max.geburtsdatum.jahr = 1992;
    Max.geburtsdatum.monat = 5;
    Max.geburtsdatum.tag = 15;

    Max.name.vorname = "Max";
    Max.name.nachname = "Mustermann";

    printf("%s ",Max.name.vorname);
    printf("%s ist geboren am "Max.name.nachname);
    printf("%d.",Max.geburtsdatum.tag);
    printf("%d.",Max.geburtsdatum.monat);
    printf("%d\n",Max.geburtsdatum.jahr);
    printf("Seine Matrikelnummer ist: %d\n");
    return 0;
}
\end{lstlisting}

\begin{tabbing}
Ausgabe: ``\=\textit{Max Mustermnn ist geboren am 15.5.1992}\\
           \>\textit{Seine Matrikelnummer ist: 335689}''
\end{tabbing}

\section{Datentypen umbenennen}

Mit dem Befehl \textbf{typedef} kann man einem Datentypen (den es bereits gibt) einen zusätzlichen Namen geben.

\begin{lstlisting}
/*der Datentyp 'int' hat jetzt
  den zweiten Namen 'GanzeZahl' */
typedef int GanzeZahl;

int main(void) {
    int i;       //erstellt 2 Variablen
    GanzeZahl j; //beide vom Typ 'int'
    return 0;
}
\end{lstlisting}

\begin{tabbing}
allgemein: (\=\textbf{T1} ist der Name eines Datentypen den es schon gibt)\\
            \>\textbf{T2} ist der neue Name, den T1 jetzt zusätzlich haben soll
\end{tabbing}
\begin{lstlisting}[numbers=none]
typedef <T1> <T2>
\end{lstlisting}

\subsection{Zusammengesetzte Datentypen umbenennen}

Genau so kann man auch Datentypen, die man selbst erstellt hat, einen neuen Namen geben.

\begin{lstlisting}
struct Datum {
    int jahr;
    int monat;
    int tag;
};

// 'struct Datum' darf jetzt auch 'date_t' genannt werden
typedef struct Datum date_t
// '_t' wird oft benutzt, wenn man Datentypen benennt

int main(void) {
    /*die Variablen 'heute' und 'morgen'
      sind vom selben Typ */
    date_t heute;
    struct Datum morgen;
    heute.jahr = 2015;
    morgen.jahr = 2015;
    return 0;
}
\end{lstlisting}

\section{Pointer auf eigene Datentypen}

Wie man auf Komponenten von selbst erstellten Datentypen zugreift haben wir eben schon gesehen. Wenn man aber nicht das Obejkt, sondern nur einen Pointer auf das Objekt hat geht das nicht.

\begin{lstlisting}[numbers=none]
date_t d;
d.jahr = 2015;
d.monat = 12;
d.tag = 15;

/* p ist vom Typ Pointer auf date_t */
date_t *p;

/* p soll auf d zeigen */
p = &d;
\end{lstlisting}

Um über die den Pointer \textit{p} auf die Komponenten zufreigen zu können, muss man ihn erst dereferenzieren (denn in der Variable \textit{p} steht nur die Adresse). Das dereferenzieren mit dem \textit{*}-Operator schreibt man am besten in Klammern und greift danach mit \textbf{.} auf die Komponenten zu.

\begin{lstlisting}[numbers=none]
if ( d.jahr == (*p).jahr ) {
    printf("alles ok\n");
} else {
    printf("ups, p scheint nicht auf d zu zeigen\n");
}
\end{lstlisting}

Statt \lstinline$(*p).jahr$ darf man auch \lstinline$p->jahr$ schreiben.\\

Ein Beispiel wie man mit \textbf{structs}, Pointer und Funktionen arbeiten kann:

\begin{lstlisting}
#include <stdio.h>

struct Datum {
    int jahr;
    int monat;
    int tag;
};

typedef struct Datum datum_t;

struct Student {
    date_t geburtsdatum;
};

typedef struct Student student_t;

/* diese Funktion vergleicht 2 Studenten anhand ihres Alters
 * sie gibt 0 zurueck wenn beide gleich alt sind
 * 1 falls der erste aelter ist
 * und -1 falls der zweite aelter ist */
int compare(student_t *s1, student_t *s2) {
    if (s1->jahr < s2->jahr) return 1;
    if (s2->jahr < s1->jahr) return -1;
    
    if (s1->monat < s2->monat) return 1;
    if (s2->monat < s1->monat) return -1;

    if (s1->tag < s2.tag) return 1;
    if (s2->tag < s1.tag) return -1;

    return 0;
}

int main(void) {
    student_t Max, Floh;
    Max.geburtsdatum.jahr = 1993;
    Max.geburtsdatum.monat = 6;
    Max.geburtsdatum.tag = 16;
    Floh.geburtsdatum.jahr = 1993;
    Floh.geburtsdatum.monat = 7;
    Floh.geburtsdatum.tag = 2;

    int result = compare(&Max, &Floh);

    if(result > 0) {
        printf("Max ist aelter\n");
    } else if (result < 0) {
        printf("Floh ist aelter\n");
    } else {
        printf("Beide sind gleich alt\n");
    }
    return 0;
}
\end{lstlisting}

\section{Enums}

Ein \textbf{enum} ist ein Datentyp dessen Instanzen sich in unterschiedlichen Zuständen befinden können. Diese Zustände kann man selbst definieren.

\begin{lstlisting}
enum Wochentag {
    MO,
    DI,
    MI,
    DO,
    FR,
    SA,
    SO
};

int main(void) {
    /*erstellt eine Variable vom Typ 'enum Wochentag' mit Namen 'heute'*/
    enum Wochentag heute;

    /* 'heute' kann sich in den Zustaenden MO, DI, MI, DO,
       FR, SO oder SO befinden */

    heute = MO; // so weist man den Wert zu

    enum Wochentag morgen = DI;
    /* Zuweisungen funktionieren genauso wie bei anderen Variablen */
    morgen = heute;
    return 0;
}
\end{lstlisting}

Enums können auch mit \textbf{typedef} umbenannt werden.

\begin{lstlisting}[numbers=none]
typedef enum Wochentag wtage_t;
\end{lstlisting}

\end{document}