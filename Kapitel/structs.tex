\documentclass[c_worksheet.tex]{subfiles}

\begin{document}
  
\chapter{Eigene Datentypen}

\section{Warum will man eigene Datentypen?}

Die standart Datentypen wie \textbf{int, float, double, char}... kennen wir schon. Wenn man sich neue Variablen von diesen Typen erstellen möchte schreibt man:
\begin{lstlisting}
int i;   //erstellt eine Variable vom Typ int
char c;  //        .    .    .           float
float f; //        .    .    .           double
\end{lstlisting}

Wenn man nun Dinge aus der realen Welt in Programmcode repräsentieren möchte, überlegt man sich aus welchen Variablen diese Dinge aufgebaut sind.\\
z.B.: \textit{Ein Datum besteht aus drei Zahlen. Jahr, Monat und Tag. Also sollte ich mir drei Variablen erstellen.} \\

Der Code dazu könnte so aussehen:
\begin{lstlisting}
int main(void) {
    //3 Variablen fuer ein Datum erstellen
    int zeitpunkt1_jahr;
    int zeitpunkt1_monat;
    int zeitpunkt1_tag;

    //3 Variablen fuer ein anderes Datum erstellen
    int zeitpuntk2_jahr;
    int zeitpuntk2_monat;
    int zeitpunkt2_tag;

    //alle Variablen mit Werten fuellen
    zeitpunkt1_jahr = 2015;
    zeitpunkt1_monat = 9;
    zeitpunkt1_tag = 15;
    zeitpunkt2_jahr = 2016;
    zeitpunkt2_monat = 3;
    zeitpunkt2_tag = 26;

    return 0;
}
\end{lstlisting}

Weil das sehr schnell sehr unübersichtlich wird (am Ende hat man tausende Variablen) darf man sich eigene Datentypen erstellen, die aus Komponenten aufgebaut sind.
Diese selbst erstellten Datentypen tragen den Namen \textbf{struct} (weil der Code durch sie strukturierter wird).

\begin{lstlisting}
//hier wird kein Befehl ausgefuehrt
//hier wird ein neuer Datentyp mit dem Name 'struct Datum' erstellt
struct Datum {   //der neue Datentyp hat 3 Komponenten
    int jahr;
    int monat;
    int tag;
};

int main(void) {
    //2 Variablen vom Typ 'struct Datum' erstellen
    struct Datum zeitpunkt1;
    struct Datum zeitpunkt2;

    //und mit Werten fuellen
    zeitpunkt1.jahr = 2015;
    zeitpunkt1.monat = 9;
    zeitpunkt1.tag = 15;

    zeitpunkt2.jahr = 2016;
    zeitpunkt2.monat = 3;
    zeitpunkt2.tag = 26

    return 0;
}
\end{lstlisting}

In den Zeilen 11 und 12 werden Variablen mit den Namen \textit{zeitpunkt1} und \textit{zeitpunkt2} erstellt. Mit dem \textbf{Punkt-Operator} kann man dann auf die Komponenten dieser Variablen zugreifen. Diese kann man genauso benutzen wie einzelne Variablen. Man kann Werte in sie hineinschreiben und Werte aus ihnen herauslesen (siehe 2.3)

\begin{lstlisting}
zeitpunkt1.jahr = 2015;
zeitpunkt2.jahr = zeitpunkt1.jahr;
\end{lstlisting}

Will man einen neuen Datentyp erstellen schreibt man allgemein:
\begin{lstlisting}
struct <name> {
    <Typ von Komponente 1> <Name von Komponente 1>;
    <Typ von Komponente 2> <Name von Komponente 2>;
      .
      .
      .
};
\end{lstlisting}

\section{Datentypen in einander schachteln}

Variablen eines selbst erstellten Datentyps können auch wieder Komponenten eines anderen Datentyps sein.

\begin{lstlisting}
#include <stdio.h>

//erstellt Datentypen mit Namen 'struct Datum'
struct Datum { // 'struct Datum' besteht aus 3 Komponenten
    int jahr;
    int monat;
    int tag;
};

/*erstellt Datentypen mit Namen 'struct Name'
  beide Komponenten sind von Typ 'Pointer auf char') */
struct Name { 
    char *vorname;   //Speicheradresse an der der Vorname liegt
    char *nachname;  //Speicheradresse an der der Nachname liegt
};

/*ein Student hat 3 Komponenten
 *Komponente 'name' ist vom Typ 'struct Name'
 *Komponente 'geburtsdatum' ist vom Typ 'struct Datum'
 *Komponente 'matrikelNummer' ist vom Typ 'int' */
struct Student {
    struct Name name;
    struct Datum geburtsdatum; 
    int matrikelNummer; 
};

int main(void) {
    struct Student Max;
    Max.matrikelNummer = 335689;
    Max.geburtsdatum.jahr = 1992;
    Max.geburtsdatum.monat = 5;
    Max.geburtsdatum.tag = 15;

    Max.name.vorname = "Max";
    Max.name.nachname = "Mustermann";

    printf("%s ",Max.name.vorname);
    printf("%s ist geboren am "Max.name.nachname);
    printf("%d.",Max.geburtsdatum.tag);
    printf("%d.",Max.geburtsdatum.monat);
    printf("%d\n",Max.geburtsdatum.jahr);
    printf("Seine Matrikelnummer ist: %d\n");
    return 0;
}
\end{lstlisting}

\begin{tabbing}
Ausgabe: ``\=\textit{Max Mustermnn ist geboren am 15.5.1992}\\
           \>\textit{Seine Matrikelnummer ist: 335689}''
\end{tabbing}

\section{Datentypen umbenennen}

Mit dem Befehl \textbf{typedef} kann man einem Datentypen (den es bereits gibt) einen zusätlichen Namen geben.

\begin{lstlisting}
/*der Datentyp 'int' hat jetzt
  den zweiten Namen 'GanzeZahl' */
typedef int GanzeZahl;

int main(void) {
    int i;       //erstellt 2 Variablen
    GanzeZahl j; //beide vom Typ 'int'
    return 0;
}
\end{lstlisting}

\begin{tabbing}
allgemein: (\=\textbf{T1} ist der Name eines Datentyoen den es schon gibt)\\
            \>\textbf{T2} ist der neue Name, den T1 jetzt zusätzlich haben soll
\end{tabbing}
\begin{lstlisting}[numbers=none]
typedef <T1> <T2>
\end{lstlisting}

\subsection{zusammengesetzte Datentypen umbenennen}

Genau so kann man auch Datentypen, die man selbst erstellt hat, einen neuen Namen geben.

\begin{lstlisting}
struct Datum {
    int jahr;
    int monat;
    int tag;
};

//an dem '_t' erkennt man, dass es ein Typname
typedef struct Datum date_t

int main(void) {
    /*die Variablen 'heute' und 'morgen'
      sind vom selben Typ */
    date_t heute;
    struct Datum morgen;
    heute.jahr = 2015;
    morgen.jahr = 2015;
    return 0;
}
\end{lstlisting}

\end{document}