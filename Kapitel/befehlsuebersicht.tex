\documentclass[c_worksheet.tex]{subfiles}

\begin{document}
	
\chapter{Befehlsübersicht} 

\section{Variablen}

In Variablen kann man Werte speichern, dafür gibt es verschiedene Typen, unter anderem:

\begin{itemize}
 	\item \textbf{int} - ganze Zahlen
 	\item \textbf{float} - Kommazahlen
 	\item \textbf{double} - Kommazahlen die doppelt so präzise sind (wie \emph{floats})
 	\item \textbf{char} - Zeichen wie z.B. 'a' oder 'b' 
 \end{itemize} 

Es gibt noch viele andere Typen für Variablen.

Variablen werden anch dem folgenden Schema angelegt

\begin{lstlisting}
	<typ> <name>;
\end{lstlisting}

und können bei ihrer \emph{Initialisierung} direkt einen Wert zugewiesen bekommen.

\lstinputlisting{CodeSnippets/Befehlsuebersicht/variablen.c} 



\end{document}